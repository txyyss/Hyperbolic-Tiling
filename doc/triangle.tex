\documentclass[12pt]{article}

\usepackage[margin=1.5cm]{geometry}
\usepackage{fontspec}
\usepackage{natbib}
\usepackage{graphicx}
\usepackage{amsmath}
\usepackage{unicode-math}

\defaultfontfeatures{Mapping=tex-text,Scale=MatchLowercase}
\setmainfont{Libertinus Serif}
\setsansfont{Libertinus Sans}
\setmonofont{Source Code Pro}
\setmathfont{Libertinus Math}


\title{Hyperbolic Tilings via Group Theory and Automata}
\author{Shengyi Wang}
\begin{document}
\maketitle

\begin{center}
  \includegraphics[width=\textwidth]{Hyp552.eps}
\end{center}

\section{Introduction}

The figure on the cover page is a hyperbolic tiling of the Poincar\'e
disk, which is a model of the hyperbolic plane. If the readers are
reading a PDF version of this article, they can zoom in the cover
figure to explore the exquisite detail on the boundary: there are more
than $179,900$ black triangles in total.

In this article I briefly introduce the mathematical principles behind
the figure and explain the algorithm which generates it. The
generation of these types of triangular tiling graphics is the
combination of geometry, group theory and deterministic finite
automata. Nothing here is new. I just want to elucidate the elegance
of mathematics hidden beneath this already stunning figure.

\section{Hyperbolic Geometry}

\end{document}
